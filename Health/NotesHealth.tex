\documentclass{article}

% Language setting
% Replace `english' with e.g. `spanish' to change the document language
\usepackage[utf8]{inputenc}
\usepackage[spanish]{babel}
\usepackage{hyperref}

% Set page size and margins
% Replace `letterpaper' with`a4paper' for UK/EU standard size
\usepackage[letterpaper,top=2cm,bottom=2cm,left=3cm,right=3cm,marginparwidth=1.75cm]{geometry}

% Useful packages
\usepackage{amsmath}
\usepackage{graphicx}
\usepackage{hyperref}
\hypersetup{
    colorlinks=true,
    linkcolor=blue,
    filecolor=magenta,      
    urlcolor=cyan,
    pdftitle={Overleaf Example},
    pdfpagemode=FullScreen,
    }

\urlstyle{same}
\title{Notas de información relevante sobre salud.}
\author{José Manuel Pedro Méndez}

\begin{document}
\maketitle

\section{Introducción}

Notas de información útil sobre que podemos hacer para mejorar nuestra Salud y controlar nuestro peso, la mayoría de nformación es principalmente obtenida de videos elaborados por personas como el doctor  \href{https://www.youtube.com/channel/UCqA4L7zgzJFS01F2km1yiDw}{LA ROSA}, \href{https://www.youtube.com/channel/UCuzmut0enwi-LrnyYwD0NCA}{Vadym Calavera},

\section{Notas sobre salud(Healh Notes)}

Para el control de peso las mejores herramientas son Control de nuestro peso son: Ayuno intermitente  y el entrenamiento de intervalos, ¿Por qué? Porque necesitamos aumentar nuestro metabolismo.

\subsection{Ayuno}

Ayunar es una herramienta nutricional, consiente, serena, con un potencia terapéutico espectacular del cual uno debe de aprender, que no se trata de comer menos, si no de comer menos seguido, \textbf{ayunar no es comer \textit{menos} en un día, \textit{AYUNAR} es comer menos \textit{SEGUIDO}}, al final de cuentas todos los seres humanos debemos ayunar pero no todos los seres humanos debemos ayunar en todos los momentos de nuestra vida, hay que evaluar: cuando si, cuando no, cuánto si y cuánto no, pero todos deberíamos hacerlo.
Pero, ¿Por qué sirve? Porque estreso ligeramente el cuerpo. 

\subsection{Comidas a evitar.}

Estás comidas bajan nuestra Testosterona.

\begin{itemize}
    \item Leche de vaca y queso(sistema endromico).
    \item Alcohol :( Además recuerda que son 140 calorías por cada 100 mililitros. 
    \item Azúcar (Nada de refresco).
    \item Exceso de Sal(Más de 2 gramos no).
    \item Aceite Vegetal.
    \item Productos de soya.
    \item Grasas Trans (margarina, comidas rápidas, donas, pasteles, papás, etc) 
    \item Plástico (bisphenola) forma peligrosa de estrógeno, evitar botellas de plástico y guardar comida en contenedores plásticos.
\end{itemize}

¿No quieres Estrógenos? 

\begin{itemize}
    \item Vegetales (brócoli y col)
    \item Hongos (crudos, con cebolla)
    \item Uvas Rojas
    \item Semillas(politinoles) de lino y fresno cesamo
    \item Cereal integral, germen, trigo, avena, centeno, maíz, arroz, mijo, cebada
    \item Te verde 
    \item Granada 
\end{itemize}

\section{No existe estar sanos de verdad sin hacer ejercicio.}

\subsection{¿Fuerza o Hipertrofía?}


Los ejercicios de fuerza se distinguen de cualquier otro tipo de entrenamiento, que es el desarrolo de la fuerza neuromuscular, existen dos procesos básicos del aumento de fuerza neuromuscular que son centrales tanto para un aumento de la espectativa de vida y la calidad de vida y para el aumento de la fuerza en sí misma, que son básicamente el reclutamiento de las neuronas motoras y la frecuencia de disparo de la estimulación de la construcción, a más rápidamente mi neurona responde mayor va a ser la capacidad de contracción del músculo y a más cantidad de neuronas musculares yo tengo reclutadas para estar trabajando en contraer el bíceps mayor va a ser la fuerza que yo pueda generar del bíceps en la contracción.\\
\\
Beneficios positivos extra de los entrenamientos de fuerza en:

\begin{itemize}
\item Mejoria del estado de ánímo.
\item Formación de nuevas neuronas, es decir, en mi desarrollo de neuronas nuevas.
\end{itemize}
¿Qué es la hipertrofía? Básicamente es mayor crecimiento y mayor desarrollo de fibras musculares, como vas a tener más fibras, vas a llevar un aumenteo de tamaño y también a un aumento de fuerza sin embargo este no es el único proceso que sucede en tu organismo cuando entrenas la hipertrofia para aumentear el volumen de tu mśsculo, sin embargo la hipertrofia aumentea el crecimiento de todos los tejidos, incluso los que no quieres que se desarrollen y muchas veces cuando tus células no funcionan correctamente se desarrola más fibra muscular pero también se desarrollan células que no quieres que se desarrollen, entonces, por un lado disminuyes tu expectativa de vida con la hipertrofía y por otro lado puedo obtener beneficios cognitivos para mi rendimiento cerebrar y para mi expectativa de vida si entreno ejercicios fuerza.\\
\\
Los ejercicios de fuerza generan más reclutamiento neuroneal, eso quiere decir que atraen neuronas motoras a trabajar en la contracción muscular del músculo que estés entrenanado, esto implica nuevas conexiones neuronales, estas conexiones no solo te hacen más fuerte, sino que te hacen más capaz cognitivamente hablando, ¿Quieres cambiar tus redes neuronales porque te no te gusta como estás pensando o porque no te gustan los hábios que tienes? Perfecto, que mejor manera de generar nuevas redes neuronales que generar conexiones nuevas, e incluso rompemos con los pensamientos respecto a lo que es posible o lo que no es posible a nivel cognitivo con los ejercicios de fuerza, es mucho más económico para tu organismo a largo plazo sistener un músculo fuerte que tiene la capacidad absoluta para hacer todos los movimientos que necesita pero que no tiene más volumen del que necesito.

Sin embargo ¿Cómo tenemos que lograr aumento de hipertrofia y cómo tenemos que lograr aumento de fuerza?\\
\\
Si yo estoy haciendo hipertrofía para un aumento del tamaño del musculo total, la evidencia es clara: Tenemos que priorizar el volumen de trabajo total, pensar en volumen de trabajo = series x repeticiones x intensidad(Peso), así +Volumen = +Hipertrofía, así que debemos aumentar la cantidad de series, repeticiones, la intensidad del trabajo que estás haciendo.
\\
\\
Es hacer:
\begin{itemize}
    \item 3-5 series
    \item 8-12 reps
    \item 70\% del peso máximo.
\end{itemize}
Ahora, si quieres empezar a trabajar fuerza como entrenamiento muscular, es que en la fuerza es mucho más importante la intensidad del entrenamiento con el que trabajamos, en general, usar pesos más altos, más cerca de nuestro máximo.\\
\\
Es hacer:
\begin{itemize}
    \item 3-5 series
    \item 2-5 reps
    \item 2 min de descanso entre series. Esto es para maximizar los niveles de testosterona que yo puedo tener obtener gracias al entrenamiento de fuerza el consejo que se da en general es alargar los periodos de descanso, y trabajar con ejercicios que incluyan grandes grupos musculares.
    \item 85\% del peso máximo.
\end{itemize}
Sin embargo, sea cual sea el método que estes empleando, lo más importante es; A más porcentaje de masa moscular y menos porcentaje de grasa, más expectativa de vida, más sensibilidad a la insulina, más cantidades de glucosa yo puedo acomodar en mi organismo sin sufrir daños, mayor densidad osea, un mayor nivel de energía y un mayor nivel de hormonas sexuales con la edad.
\subsection{¿Cómo entrenar para ganar músculo?}
Lo primero que hay que saber es que tu cuerpo construye y destruye músculos. Es decir:
\begin{itemize}
    \item Si construyo más músculo del que construyo voy a ganar masa muscular.
    \item Si destruyó más músculo del que construyo voy a perder masa moscular.
\end{itemize}
Tomar aminoacidos nos ayuda a construir músculo, recuperarnos más rapido del entrenamiento impidiendo la destrucción del músculo.\\
\\
Nuestro cuerpo, en realidad se adapta al esfuerzo que estas haciendo. Por eso es más facíl digerir alimentos que llevamos comiendo toda nuestra vida comparativamente con alimentos que empezamos a comer ayer.\\
\\
El desarrollo del músculo nuevo déspues de entrenar sucede solamente cuando terminaste de reparar el daño de las fibras musculares que se rompieron por el entrenamiento. Es decir, construyo músculo sólo cuando no estoy adolorido luego de entrenar, mientras más rápido me recupero más rápido constuyo músculo nuevo, la keratina y los aminoácidos asenciales ayudan a acelerar este proceso.\\
\\
¿Cómo optimizar el desarrollo muscular?
\begin{enumerate}
    \item Hacer más series estimulan más al músculo que realizar una sola serie. 
    
    \item A mayor volumen semanal entrenamiento mayor ganancia de masa muscular.
\end{enumerate}
Aumenta el volumen de trabajo del músculo que quiero entrenar, EVITA el daño excesivo, no importa la cantidad de repeticiones, lo importante es agotar el músculo llegando casi al fallo(El momento donde ya no puedo levantar el peso), APUNTO de llegar PERO NO llegar.\\
\\
Excederse en proteína no lo vale, nuestro cuerpo solo puede construir por día una cantidad de músculo establecida sin importar nuestro tamaño. 20 gramos de proteína por comida dan un desarrollo muscular casi máximo, el EXCESO de proteína es MALO.\\
\\
¿Qué proteína me sirve más para el desarrollo muscular? 
\begin{itemize}
    \item Leucina (Aminoácido que necesitamos para sintetizar proteína).
    \item Aminoácidos esenciales + Aminoácidos raminificados(ricos en leucina)
    \item NO consumir leche.
\end{itemize}
No debemos consumir alimentos excesivos en azúcar y grasa, es un mito lo de que podemos comer de todo cuando hacemos ejercicio.\\
\\
¿Ayuno y ejercicio? Si, ya que entrenar y hacer ayuno intermitente NO reduce nuestra masa muscular, de hecho, el ayuno ideal: 16/8 -> 16 horas de ayuno y 8 horas de ingesta.\\
\\
Si tienes baja presión o te mareas cuando haces ejercicio te conviene hacerlo en las horas de comida, si quieres bajar de peso conviene entrenar a penas comienza el ayuno porque aceleras tu metabolismo y aceleras la lipolisis, si quieres aumentar de masa muscular conviene entrenar en las horas de ayuno habiendo consumiendo keratina y aminoácidos durante el ejercicio y comer luego de este trabajo muscular y para el resto de nosotros es ideal entrenar en el periodo de ayuno.\\
\\
¿Cúanto tiempo debo descansar entre cada serie? 3-5 min de descanso entre serie es lo ideal para maximizar el desarrolo muscular, un periodo mayor de descanso entre series, estimula la producción de músculo luego del ejercicio, y esto sucede porque a mayor descanso tengo menos fátiga, es decir, menos daño muscular y NOS CONVIENE NO TENER DAÑO MUSCULAR, Y si tu objetivo es desarrollar un músculo sano y fuerte te conviene entrenar más días a la semana, pero con menos intensidad.\\
\\
¿Cada cúando entrenar? Con un solo día de ejercicio muscular puede estimular la generación por hasta 72 horas pero el momento donde más estoy estimulando el desarrollo muscular se encuentra 24 hrs después del entrenamiento así entrenar al menos 2 veces por semana cada grupo muscular esta bien.\\
\subsection{¿Cómo y por qué crecen nuestros músculos?}

Primero que nada, el entrenamiento debe de ser acorde a las características de cada persona, una persona ectomorfa no necesita el mismo tipo de entrenamiento que una persona que siempre fue endomorfa.\\
\\
La actina y la miosina se acercan o alejan entre si cuando generamos la contracción múscular, A mayor peso levantamos mayor es la contracción entre ambas partes. Por otro lado existe una proteína llamada Proteína Titán, esta proteína es la responsable de la elasticidad pasiva del músculo.\\
\\
Cada músculo tiene un peso ideal al que se puede ser sometido para crecer. Si trabajamos con menos peso necesito levantar el peso durante más tiempo para obtener el mismo desarrollo muscular que cuando trabajamos con el peso máximo, es decir mantener al músculo tensionado por más tiempo.\\
\\
Cúando entrenamos un músculo, a mayor peso utilizamos, mas repeticiones o mayor frecuencia de trabajo semanal realizamos, mayor será va a ser el crecimiento proporcionado que vamos a tener(Desarrollo muscular), un fragmento de la proteína titán indica a las células la necesidad de construir músculo. Y lógicamente, a más tensión sometemos el músculo o a más tiempo lo sometemos a esa misma tensión, más se va a señalizar a tus células que el músculo tiene que desarrollarse y por ende más va a crecer, sin embargo la proteína llamada titán no solo se estimula con la contracción, también se estimula cuando el músculo se estira. Los ejercicios excéntricos son más útiles para el crecimiento muscular.\\
\\
Sin embargo, ¿Cómo estimular el crecimiento muscular? $70\%$ del peso total que podemos levantar en un movimiento parece ser el ideal para desarrollar nuevo músculo, pero lo interesante es que trabajar por dejabo del $70\%$ de peso ideal reduce considerablemente el efecto de la proteína titán, esto impide que nuestras células señalicen la necesidad de construir más músculo mientras que trabajar por encima del $70\%$ de peso ideal agota más rápido nuestros músculos alterando el funcionamiento de la proteína.\\
\\
Aumentar progesivamente el peso que levantamos o la dificultad de los movimientos que haces cuando entrenas con tu cuerpo, es esencial, para el desarrollo muscular. \\
\\
Por lo qué: \textbf{¿Cómo entrenar?} 
\begin{enumerate}
    \item Tener la cantidad de ATP necesarioa para entrenar, entrena durante tus horas de ingesta para mejorar tu desarrollo muscular, no lo hagas durante el ayuno, utiliza suplementos como creatina para darle a tus músculos los recursos necesarios.
    \item Entrena con el $70\%$ del peso máximo que puedas levantar. (PARA HIPERTROFÍA, no para FUERZA).
    \item Es recomendado entrenar grupos musculares cada 48 horas. \textbf{RECUERDA:} que tu cuerpo empieza a construir músculo cuando termina de reparar el daño, por lo que no te exedas de daño generado. Te conviene dividir tu entrenamiento en grupos musculares. 
    \item Incorpora ejercicios excéntricos  
    \item Consumir al menos 2.5 díarios de leucina para ayudarte a construir músculo. 
\end{enumerate}

\section{¿Extras? Posiblemente tonterías}
\subsection{TU BELLEZA DEPENDE DE TUS HÁBITOS, no de tus genes}
Tenemos 14 huesos en nuestra cara y esos huesos no están funcionando entre sí y muchas cosas a lo largo de nuestra vida puede hacer nuestra cara asímetrica.
simetría.\\
\\
Nosotros podemos mejorar nuestra simetría facial ya que los huesos de la cara no se fucionan entre sí hasta una avanzada edad, por lo que podemos conservar una pequeña cantidad de movimiento durante la mayoría de nuestra vida.\\
\\
Podemos alterar nuestra simetría de nuestra cara si: 
\begin{itemize}
    \item Fumando (En un estudio en gemelos notamos que los que fumaban o tenían problemas dentaríos tenían más canteo del plano oclusual).
    \item Por alteración postural.
    \item Por problemas dentarios.
    \item Por diferente visión en los ojos.
\end{itemize}

¿Que podemos cambiar? El desarrollo y la tensión muscular son esenciales para la simetría facial.
 \begin{enumerate}
     \item Masticar más permite el desarrollo de ut mandíbula (Con comida real, no procesada).
     \item Una dieta de bajo índice glúcemico cuida tus dientes y a la flora bacteriana de tu boca.
     \item Una mala posición de la lengua trae problemas estéticos y de salud, corregir la postura de lengua nos ayuda a respirar mejor y por nariz, para corregir una mala posición de la lengua pordemos utilizar la técnica Mewing.
     
     \item Evitar las alergías respiratorias ya que al mantener la boca abierta para respirar impedimos la posición correcta de la lengua, si respiras por la boca podemos hacer ejercicios de método:  \href{https://www.anahana.com/es/breathing-exercise/buteyko-breathing}{buteyko}).
     \item A peor flora bacteriana tenemos mayor posibilidad de alergías y enfermedades autoinmunes, por ende debemos mantener una flora bacteriana sana y para ello lo más simple que podemos hacer podemos usar prebióticos
 \end{enumerate}\\
Una cara símetrica  nos hace más saludables y nos da mayor confianza\\
\\
Más detalles: \url{https://www.youtube.com/watch?v=ts0s80GFg-g}
We have fourteen bones in our head and this have m
\end{document}